\documentclass{article}% ===> this file was generated automatically by noweave --- better not edit it
\usepackage{amsmath, amsfonts, amssymb}
\usepackage{amsthm}
\usepackage{noweb}

\theoremstyle{plain}
\newtheorem{theorem}{Theorem}[section]
\newtheorem{corollary}[theorem]{Corollary}
\newtheorem{lemma}[theorem]{Lemma}
\newtheorem{conjecture}[theorem]{Conjecture}
\newtheorem{proposition}[theorem]{Proposition}
\newtheorem{construction}[theorem]{Construction}

\theoremstyle{plain}
\newtheorem*{claim}{Claim}
\newtheorem*{program}{Program}

\theoremstyle{definition}
\newtheorem{definition}[theorem]{Definition}
\newtheorem{definitions}[theorem]{Definitions}
\newtheorem{notation}[theorem]{Notation}
\newtheorem{example}[theorem]{Example}
\newtheorem{examples}[theorem]{Examples}
\newtheorem{remark}[theorem]{Remark}
\numberwithin{equation}{section}

\noweboptions{smallcode,longchunks}
\begin{document}
\title{{\Tt{}coover\nwendquote}: glue between {\Tt{}heegaard\nwendquote} and {\Tt{}twister\nwendquote}}
\author{Robert C. Haraway, III}
\date{\today}
\maketitle

\section{Introduction}
The following problem
has come to be important to a project I'm working on:
\begin{quotation}
Given a finite presentation $P$ of a group $G$, to
construct a triangulation of a 3-manifold $M$ with
fundamental group $G$.
\end{quotation}
There are various issues with this problem, not least
of which is the undecidability of 3-manifoldness
(indeed, even of triviality) from a finite presentation.
For the purposes of the project, it will suffice to
solve the following less impossible problem:
\begin{quotation}
Given a finite presentation $P$ of a group $G$,
to determine whether or not $P$ is the associated presentation
of a (generalized) Heegaard splitting of some 3-manifold, and if so,
to construct a triangulation of the 3-manifold so split.
\end{quotation}
John Berge's program {\Tt{}heegaard\nwendquote}\cite{heeg} can determine
realizability of a presentation as a Heegaard splitting, and
construct such a splitting if one exists. But it cannot
construct triangulations.

Mark Bell, Tracy Hall, and Saul Schleimer wrote a program
{\Tt{}twister\nwendquote} that can take as input a surface $S$ and two attaching
multicurves $\gamma_0,\gamma_1$, represented as a cubulation and two sets of disjoint
hyperplanes, and triangulate the result of attaching
2-handles to $S \times [0,1]$ along $\gamma_0\times\{0\}$ and $\gamma_1\times\{1\}$.
If $S$ has genus $g$ and $\gamma_0$ has $g$ components, then this is
a generalized Heegaard splitting where the components of $\gamma_0$
are boundaries of compressing discs for the starting handlebody.

All that remains to solve the problem, then, is to turn
the output of {\Tt{}heegaard\nwendquote} into a cubulation and list of hyperplanes.

I thank Mark Bell and Saul Schleimer 
for helpful conversations
on how this could be accomplished.

\section{Digestion and its results}
The following is a list of presentations relevant
to the aforementioned project.
\nwfilename{coover.nw}\nwbegincode{1}\moddef{heeg.in}\endmoddef\nwstartdeflinemarkup\nwenddeflinemarkup
0
mGGmgNg
MNmn

\nwendcode{}\nwbegindocs{2}\nwdocspar
To run {\Tt{}heegaard\nwendquote} on this input in a
POSIX operating system,
one places this file in the directory
in which the {\Tt{}heegaard\nwendquote} executable
is located; then one renames
it {\Tt{}Input{\_}Presentations\nwendquote}; then one runs {\Tt{}heegaard\nwendquote}
from a command line; then one selects
option `B'; then one selects `d'; then
one replies `n` to further options. A session
with {\Tt{}heegaard\nwendquote} following these directions
went as below:
\nwenddocs{}\nwbegindocs{3}\nwdocspar
\nwenddocs{}\nwbegincode{4}\moddef{heegaard.out}\endmoddef\nwstartdeflinemarkup\nwenddeflinemarkup
$ ./heegaard 


                                 HEEGAARD
                               BY JOHN BERGE
                             jberge@charter.net
                                   2/5/15

 A PROGRAM FOR STUDYING 3-MANIFOLDS VIA PRESENTATIONS AND HEEGAARD DIAGRAMS.

        Copyright 1995-2015 by John Berge, released under GNU GPLv2+.

Note! Open the file 'Heegaard_Diagrams.dot' in Graphviz() to see Heegaard's Heegaard diagrams.

    Note!   Hit the 'space-bar' to interrupt long computations.
            Hitting 's' during long computations should provide a status report.

HIT 'B' TO DO SOME BATCH PROCESSING.
HIT 'f' IF THE PRESENTATION WILL COME FROM THE FILE 'Input_Presentations'.
HIT 'k' IF THE PRESENTATION WILL BE ENTERED FROM THE KEYBOARD.
HIT 'q' TO QUIT RUNNING THE PROGRAM.


Hit 'a' TO COMPUTE ALEXANDER POLYNOMIALS OF 2-GENERATOR 1-RELATOR PRESENTATIONS.
HIT 'b' TO FIND 'meridian' REPS M1 & M2 OF 2-GENERATOR 1-RELATOR PRESENTATIONS.
    (This allows one to check if such presentations are knot exteriors.)
HIT 'c' TO CHECK REALIZABILITY OF PRESENTATIONS.
HIT 'C' TO CHECK IF THE INITIAL PRESENTATION IS A "HS REP". (Heegaard will stop and alert the user if
    a sequence of handle-slides of the initial presentation P yields a presentation P' with |P'| < |P|.)
HIT 'd' TO SEE DATA FOR HEEGAARD DIAGRAMS OF PRESENTATIONS.
HIT 'D' TO SEE THE DUAL RELATORS OF EACH REALIZABLE PRESENTATION'S DIAGRAM.
HIT 'E' TO HAVE HEEGAARD STABILIZE THE IP, COMPUTE HS REPS AND CHECK IF THE IP APPEARS ON THE HS LIST.
HIT 'h' TO FIND THE INTEGRAL FIRST HOMOLOGY OF PRESENTATIONS.
HIT 'l' TO FIND THE SIZE OF ORBITS OF PRESENTATIONS UNDER LEVEL TRANSFORMATIONS.
HIT 'q' TO QUIT RUNNING IN BATCH MODE.
HIT 'r' TO REDUCE AND SIMPLIFY PRESENTATIONS USING DEPTH-FIRST SEARCH AND SEP_VERT SLIDES.
HIT 'R' TO REDUCE AND SIMPLIFY PRESENTATIONS USING BREADTH-FIRST SEARCH AND SEP_VERT SLIDES.
HIT 's' TO FIND SYMMETRIES OF PRESENTATIONS.
HIT 'S' TO CONVERT SNAPPY FORMAT PRESENTATIONS TO HEEGAARD READABLE PRESENTATIONS.
HIT 'u' TO STABILIZE PRESENTATIONS WHILE PRESERVING REALIZABILITY.
HIT 'x' TO SIMPLIFY PRESENTATIONS BY SUCCESSIVELY DELETING PRIMITIVES, WITHOUT CHECKING REALIZABILITY.
HIT 'X' TO FIND PRESENTATIONS OBTAINED BY DELETING PRIMITIVES FROM ONLY INITIAL PRESENTATIONS.
HIT 'z' TO REDUCE PRESENTATIONS TO MINIMAL LENGTH.


SHOW WHICH FACES OF EACH DIAGRAM FORM BDRY COMPONENTS ?  HIT 'y' OR 'n'.

PRINT DUAL RELATORS OF EACH DIAGRAM ? HIT 'y' OR 'n'.

PRINT PATHS CONNECTING FACES OF EACH DIAGRAM ? HIT 'y' OR 'n'.

------------------------------------

0
    mGGmgNg
    MNmn

L 11, Gen 3, Rel 2.

Rewrote the presentation using the substitution: CbA.

 The rewritten presentation is:
    AABaCaB
    BCbc


I) The following table gives the number of edges joining each pair of vertices.

(A --> a1,B1,b1,C1), (a --> B1,b1,c1), (B --> C1,c1), (b --> C1,c1)

For each (X,x) pair of vertices with (X,x) = (A,a), (B,b) ... ,(Z,z):
1) Number the edges at vertex X counter-clockwise about vertex X giving the 'first-edge' at vertex X number 0.
2) Note the 'first-edge' at vertex V is the first edge in counter-clockwise order about V which connects V to V's 'first-vertex'. (See III below for V's 'first-vertex'.)
3) For x = a,b ... ,z, number the edges at vertex x clockwise about x, giving the 'first-edge' at x the number shown in the following list:

(1,0,0)

II) Vertices in the boundary of each face of the Heegaard diagram in clockwise order are:

F1) AaB, F2) ABC, F3) ACb, F4) Aba, F5) abc, F6) acB, F7) BcbC

Note: Heegaard chose the cycle 'BcbC' to be the boundary of the 'infinite' face.

III) CO[v] lists the vertices in the link of vertex v in counter-clockwise cyclic order starting with the 'first-vertex' in lexicographic order connected to v.

CO[A] = abCB, CO[a] = ABcb, CO[B] = ACca, CO[b] = AacC, CO[C] = AbB, CO[c] = aBb

****************************************************************************************
The following lines describe the Heegaard diagram in Graphviz() readable form.
Copy and paste into 'Heegaard_Diagrams.dot to have Graphviz() display the diagram.

 graph G\{layout = neato; model = circuit; size = "10.0,8.0"; ratio = fill ;
 label = "Diagram of Presentation 1 of the Initial Presentation 0"; 
node [shape = circle, fontsize = 10, height = 0.1, style = white] 
A [pos = "240,262!"]; a [pos = "339,187!"]; B [pos = "30,30!"]; b [pos = "550,420!"]; C [pos = "30,420!"]; c [pos = "550,30!"]; 
 edge [fontsize = 10]; \{ A -- a ; A -- B ; A -- b ; A -- C ; a -- B ; a -- b ; a -- c ; B -- C ; B -- c ; b -- C ; b -- c ; \}\}




 Totals: NumPresExamined 1

HIT 'B' TO CONTINUE IN 'BATCH' MODE. HIT 'q' TO QUIT RUNNING IN BATCH MODE.

HIT 'B' TO DO SOME BATCH PROCESSING.
HIT 'f' IF THE PRESENTATION WILL COME FROM THE FILE 'Input_Presentations'.
HIT 'k' IF THE PRESENTATION WILL BE ENTERED FROM THE KEYBOARD.
HIT 'q' TO QUIT RUNNING THE PROGRAM.



\nwendcode{}\nwbegindocs{5}\nwdocspar
It turns out that the only information we need from
{\Tt{}heegaard\nwendquote} is, for each presentation $P$,
\begin{itemize}
\item the presentation that {\Tt{}heegaard\nwendquote} transforms $P$ into;
\item the table from part I;
\item the tuple from part I; and
\item the string array {\Tt{}CO\nwendquote} from part III.
\end{itemize}

Culling the extra output is an exercise in regular
expressions. The following is a sequence of {\Tt{}sed\nwendquote}
regular expressions that accomplishes this
for the above output:
\nwenddocs{}\nwbegincode{6}\moddef{cull.sed}\endmoddef\nwstartdeflinemarkup\nwenddeflinemarkup

\nwendcode{}\nwbegindocs{7}\nwdocspar
After this, we need to rewrite this all in a format
that can be interpreted in Python (for {\Tt{}twister\nwendquote} is
a Python module). The following {\Tt{}sed\nwendquote} script does this.
\nwenddocs{}\nwbegincode{8}\moddef{rewrite.sed}\endmoddef\nwstartdeflinemarkup\nwenddeflinemarkup

\nwendcode{}\nwbegindocs{9}\nwdocspar
The result is the following file of data ready
for input to Python.
\nwenddocs{}\nwbegincode{10}\moddef{digested.py}\endmoddef\nwstartdeflinemarkup\nwenddeflinemarkup

\nwendcode{}\nwbegindocs{11}\nwdocspar
\section{An example worked by hand}
We now present a hand-worked example to motivate what follows.
The following is the output of {\Tt{}heegaard\nwendquote} digested for the $1^{st}$
presentation Andrew submitted (not the $0^{th}$ but the $1^{st}$):
\nwenddocs{}\nwbegincode{12}\moddef{example}\endmoddef\nwstartdeflinemarkup\nwenddeflinemarkup
heeg_data =
(['AABACaabac', 'BCbc'],
 \{'A':\{'a':2, 'b':2, 'C':1, 'c':1\},
  'a':\{'B':2, 'C':1, 'c':1\},
  'B':\{'C':1, 'c':1\},
  'b':\{'C':1, 'c':1\}\},
 [5,1,0],
 \{'A':'acbC',
  'a':'ACBc',
  'B':'aCc',
  'b':'AcC',
  'C':'AbBa',
  'c':'AaBb'\})
\nwendcode{}\nwbegindocs{13}\nwdocspar
Its type is
\begin{verbatim}
([Relator],
 UndirectedIncidenceMatrix,
 [FlagName],
 {Generator:NeighborString}).
\end{verbatim}
\nwenddocs{}\nwbegindocs{14}\nwdocspar
In turn,
\begin{itemize}
\item {\Tt{}Generator\ ::=\ Char\nwendquote} (indeed, {\Tt{}heegaard\nwendquote}'s
      generators are alphabetic ASCII {\Tt{}Char\nwendquote}s with
      {\Tt{}swapcase\ =\ inverse\nwendquote});
\item {\Tt{}Relator\ ::=\ [Generator]\nwendquote}, a {\Tt{}List\nwendquote} of {\Tt{}Generator\nwendquote}s,
      i.e. {\Tt{}Relator\ ::=\ String\nwendquote};
\item {\Tt{}Undirected{\_}Incidence{\_}Matrix\ ::=\ {\nwlbrace}Generator\ :\ {\nwlbrace}Generator\ :\ Int{\nwrbrace}{\nwrbrace}\nwendquote};
      the intended semantics is that if {\Tt{}inc\nwendquote} is this object for
      a presentation with Whitehead graph $G$, and $v,w$ are vertices of
      $G$, and $v <_{\alpha} w$, then {\Tt{}inc[v][w]\nwendquote} is the number of edges
      between $v$ and $w$;
\item {\Tt{}Flag{\_}Name\ ::=\ Int\nwendquote}; suffice to say for now that a \emph{flag}
      is an incident pair of vertex and edge in the Whitehead graph; and
\item {\Tt{}NeighborString\ ::=\ [Generator]\nwendquote}; the intended semantics of this is
      a rotation system for the reduced Whitehead graph.
\end{itemize}
\nwenddocs{}\nwbegindocs{15}\nwdocspar
Berge's documentation \cite{heeg}
explains what Whitehead graphs are, reduced or otherwise.
Lando and Zvonkin have written an entire monograph \cite{LZ} on rotation
systems, i.e. on cellular graph embeddings on surfaces, i.e.
on dessins d'enfants. Wikipedia has a quick summary of this subject.
\nwenddocs{}\nwbegindocs{16}\nwdocspar
Briefly, a rotation system is a pair $(\sigma, \alpha)$ of a
\emph{somme}- or vertex-permutation $\sigma$ and an \emph{ar\^{e}te}-
or edge-permutation $\alpha$ on a set $\Phi$ whose elements are called
\emph{flags} or \emph{darts}, such that $\alpha^2 = e$,
such that $\alpha$ fixes no flag, and such that $\langle \sigma, \alpha \rangle$
acts transitively on $\Phi$. (One may also have a face-permutation $\phi$
such that $\sigma \alpha \phi = e$.)
\nwenddocs{}\nwbegindocs{17}\nwdocspar
There is a natural bijection between rotation systems and labelled embeddings
of graphs in surfaces whose complements' components are contractible (i.e. are 2-cells).
\nwenddocs{}\nwbegindocs{18}\nwdocspar
The Whitehead graph of a Heegaard diagram is a graph embedded on the sphere.
One might hope to represent this embedding as a rotation system.
A planar embedded graph admits a representation as a rotation system precisely
when it is connected.
\nwenddocs{}\nwbegindocs{19}\nwdocspar
When a Heegaard diagram's Whitehead graph admits a rotation system representation,
one may represent the entire Heegaard diagram itself with a modicum of
additional information, as follows. Let us regard the vertices of the Whitehead
graph as being discs in the boundary of a 3-ball, such that when one glues
up discs of inverse vertices, one gets the base handlebody. Let $q$ be the
quotient map from the ball to the handlebody. 
We may regard the flags of the Whitehead graph as the union of
preimages $q^{-1}.(\gamma \cap q.D)$ over all attaching curves $\gamma$
in the Heegaard diagram and all vertices $D$ of the Whitehead graph.
To determine the Heegaard diagram, it suffices to pick for every vertex $D$ a choice of flag
$f.D$ such that for every pair of inverse vertices $d,D$, $q.(f.d) = q.(f.D)$.
\nwenddocs{}\nwbegindocs{20}\nwdocspar
Conversely, every flag-choice determines some Heegaard diagram with the given
Whitehead graph.
\nwenddocs{}\nwbegindocs{21}\nwdocspar

\nwenddocs{}\nwbegindocs{22}\nwdocspar
\section{Extraction: the core}
\subsection{Neighbor logging suffices...}
\subsection{The inner loop: or, the next flag}
\subsection{...but we can do better: naming cycles}
\section{Extraction: the rest}
\begin{thebibliography}{9}
\bibitem{heeg}
Berge, John. {\Tt{}heegaard\nwendquote}. Available at 
\texttt{http://www.math.uic.edu/t3m/}.
\bibitem{vG}
van Gasteren, A. J. M. \emph{On the shape of mathematical arguments.}
With a foreword by Edsger W. D{\ij}kstra. \emph{Lecture Notes in
Computer Science}, \textbf{445}. Springer-Verlag, Berlin, 1990.
\end{thebibliography}
\nwenddocs{}\nwbegindocs{23}\nwdocspar
\end{document}
\nwenddocs{}
